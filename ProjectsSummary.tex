\documentclass{article}
\pagenumbering{gobble}

\usepackage{hyperref}
\hypersetup{colorlinks=true,urlcolor=blue}

\usepackage[left=1.5in, right=1.5in]{geometry}

\newcommand{\header}[2]{
    \par\noindent\Large\textbf{#1}\normalsize\newline
    \url{#2}\newline
}

\title{Summary of Projects}
\author{Berk Unver}
\date{October 2023}


\begin{document}
\maketitle

\header{Work-in-Progress Game}{https://github.com/BerkUnver/UnannouncedGame}
This is a game I have been working on since December 2020. It is what propelled me to study Computer Science. It is written in C\# with the help of Godot Game Engine. Because I plan to release this, it is closed-source. The linked Github repository contains a lengthier reflection on this project.
\\\\

\header{Compiler}{https://github.com/BerkUnver/Creed}
This is a partial compiler made as the term group project for my Compilers course, written in C. I implemented the lexer, parser, and was working on the typechecker when the semester ended. I chose to implement the lexer and parser myself instead of using an external parser-generator tool as a learning experience.
\\\\

\header{Animation Metadata Editor}{https://github.com/BerkUnver/CombatAnimator}
This is a simple editor I made to add metadata to animations for my game. I needed an easy way to add information about attacks, including hitboxes and hurtboxes, to sprite animations. I chose to make this in C to practice the language. It exports JSON files that I can play back using a custom animation player in the game.
\\\\

\header{Web Assembly Interpreter}{https://github.com/BerkUnver/Zapper}
This is an interpreter for a subset of Web Assembly written in the purely-functional language Elm. It was the result of a semester-long independent study project in functional programming for course credit. I had been interested in functional programming for some time leading up to this project, and it gave me a nontrivial use-case to test my learning.

\end{document}
